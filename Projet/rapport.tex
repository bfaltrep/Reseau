\documentclass[a4paper, 12pt]{report}
\usepackage[utf8]{inputenc}
\usepackage[french]{babel}
\usepackage[T1]{fontenc} % permettent d'utiliser tous les caractères du clavier.
\usepackage{inputenc} % permettent d'utiliser tous les caractères du clavier.
\usepackage[top=2.0cm, bottom=2.0cm, left=2.0cm, right=2.0cm]{geometry} %marges des pages
\usepackage{color}
\usepackage{fancyhdr} %apparence globale
\usepackage {hyperref} %table des matières avec liens ds le dossier.
\usepackage{graphicx} %image
\usepackage{setspace}


\definecolor {colortitre1}{rgb}{0.2,0.4,0.2}
\definecolor {colortitre2}{rgb}{0.2,0.5,0.2}
\definecolor {colortitre3}{rgb}{0.2,0.6,0.1}
\pagestyle{plain}

\title{Le Voyageur de Commerce}
\author{\textbf{IN400A4 Groupe G}\\ FALTREPT Bérénice \\ SCHOETERS Jason \\ SOW Sokhna Maimouna}
\date{année 2013/2014}

\begin{document}
	\begin{spacing}{1.3}
\maketitle%sert a afficher le title dans le document, sinon serait des méta données.
\tableofcontents
\newpage
\textcolor{colortitre1}{\section*{Présentation}} \addcontentsline{toc}{section}{Présentation}

notre projet est centré sur un aquarium distribué. L'objectif étant de que les items mobiles de l'aquarium soient dépendant d'un aquarium créateur mais visibles sur les divers ordinateurs reliés au serveur.

\textcolor{colortitre1}{\subsection*{Les Modules}}  \addcontentsline{toc}{subsection}{Les Modules}

Nombre de classes indispensables à ce projet nous ayant été fournis, nous n'allons pas les décrire en détail ici.

	\textcolor{colortitre2}{\subsection*{Les classes statiques}} \addcontentsline{toc}{subsection}{Les classes statiques} 
	
		\textcolor{colortitre3}{\subsection*{Main}} \addcontentsline{toc}{subsection}{Main} 
		
Cette classe sert, comme son nom l'indique, au lancement de l'exécution du programme. elle permet notamment de tenir compte des arguments donnés par l'utilisateur pour lancer soit un serveur, soit un thread client.
		
		
		\textcolor{colortitre3}{\subsection*{Constante}} \addcontentsline{toc}{subsection}{Constante} 
Cette classe est une énumération des commandes acceptées par notre protocole. Elle permet d'assurer la conformité des commandes qui seront passées, mais surtout de rendre le code du protocole plus lisible.

	\textcolor{colortitre2}{\subsection*{Les classes instanciables}} \addcontentsline{toc}{subsection}{Les classes instanciables} 	

		\textcolor{colortitre3}{\subsection*{ServerThread}}    \addcontentsline{toc}{subsection}{ServerThread}

	Le ServerThread est la classe instancié si l'utilisateur lance un serveur via le main. \\
	Elle créée les variables nécessaires au contact avec les clients mais aussi à la gestion de l'aquarium initial du serveur. \\
	Pour s'assurer que le nombre de clients ne posera pas de problèmes durant l'exécution, nous utilisons la classe \textit{ScheduledExecutorService} qui va créer 10 Threads. Ces Threads traiteront les envois et réceptions des sockets des clients, quel que soit leur nombre. 

		\textcolor{colortitre3}{\subsection*{}}    \addcontentsline{toc}{subsection}{}
		
		\textcolor{colortitre3}{\subsection*{Aquarium}}    \addcontentsline{toc}{subsection}{Aquarium}
\textcolor{colortitre1}{\section*{}} \addcontentsline{toc}{section}{}

	\textcolor{colortitre2}{\subsection*{}}    \addcontentsline{toc}{subsection}{}


\textcolor{colortitre1}{\section*{}} \addcontentsline{toc}{section}{}

\textcolor{colortitre1}{\section*{Conclusion}} \addcontentsline{toc}{section}{Conclusion}

\newpage
\textcolor{colortitre1}{\section*{Annexe}}   \addcontentsline{toc}{section}{Annexe}

	
	
	
	\end{spacing}
\end{document}





